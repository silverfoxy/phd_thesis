\section{Discussion}
In this section, we discuss some design decisions when building \animatedead{}. 

\paragraph{Path condition analysis:} 
Concolic execution engines often incorporate SMT solvers such as Z3~\cite{moura2008z3} to evaluate the feasibility of symbolic branches based on path constraints and replace symbolic variables with concrete samples. 
In this work, to reduce the complexity of our operations and reduce the overhead of SMT solvers, we opted to implement our context-specific concolic translator. 
As discussed in Section~\ref{sec:concolic_translation}, \animatedead{} tracks path constraints for strings, set operations (i.e., array membership checks and value set analysis), and variable types. 
Using this information, the emulator skips non-feasible branches and replaces symbolic variables with concrete values only when required (i.e., instructions that introduce new code to the AST). 

\paragraph{Runtime threshold and termination condition:} 
The decision to terminate a concolic execution campaign can determine the completeness of its results. 
Premature termination can lead to missed code-coverage and in turn, false positives in debloating. 
The two mainstream approaches to this problem are time-based thresholds (e.g., terminate the analysis after \emph{N} hours of no new discovery) or missing mass estimators~\cite{goodestimator}. 
The premise of an estimator is to determine, given the prior inputs and their unique code-coverage, what is the probability of the next input to explore new code-coverage. 

In this work, we opted to use a threshold of 24 hours based on our experiments. 
In practice, only a subset of all entry points in larger web applications require runtime of more than 1 hour. 
\animatedead{} provides a reporting panel that plots the new code-coverage information for each running campaign. 
We suggest that analysts oversee the convergence of new code-coverage discovery and terminate the analysis of each entry point after concolic execution stops identifying new code-coverage for an extended period of time. 

\paragraph{Single Page Applications (SPA):} 
These web applications incorporate JavaScript libraries and asynchronous requests to interact with the web server and navigate through the web application seamlessly. 
\animatedead{} operates based on web server logs, as a result, the client side intricacies of SPAs do not affect its analysis. 

% \paragraph{Concolic execution of other languages:} 
% We built \animatedead{} based on the PHP language, which is the most popular web development language used by more than 77.4\% of all websites~\cite{w3techphp}. 
% Majority of the development effort of this project is targeted towards the PHP language. 
% As a result, for \animatedead{} to support other languages, one needs to develop new concolic execution engines. 
% That said, given the similarity of web programming languages (e.g., dynamic constructs, use of callbacks, etc.), we envision that they can benefit from the design concepts of \animatedead{}. 
