\section{Background}
Program analysis historically incorporates static analysis, dynamic analysis, as well as a hybrid of both. 
While static analysis systems are easier to run at scale, building static analysis tools that support all features within a language is difficult, and even unnecessary in many use cases. 
As a result, it is a common practice to build context-specific static analysis tools by limiting the scope of analysis (e.g., intra-procedural dataflow analysis). 

One of the main limitations of static analysis when it comes to debloating is its inability to perform end-to-end program analysis due to the presence of dynamic code structures. 
An analysis of popular PHP static analysis tools showed that even the state-of-the-art tools fail to analyze more than 20 consecutive instructions before encountering an unsupported code structure~\cite{altestability}. 
% This limitation is exacerbated when such tools are employed for an end-to-end web application analysis which is a requirement for debloating. 
The failure to resolve the dynamic code structures (e.g., file inclusions, dynamic function calls, etc.) results in misjudging the reachability of the required modules. 
Removing such modules causes false positives. 
In other words, the debloated web application will miss files and functions that are required by the users resulting in breakage. 

On the other hand, dynamic debloating schemes such as Less is More and \dbltr{} rely on an extensive training phase during which all the desired features need to be exercised. 
The slightest oversight in the exercised features will result in removal of features that are necessary. 
In Chapter~\ref{chap:lim}, we built synthetic test cases to model the user behavior. 
An alternative is to perform the training phase on real users by instrumenting live web servers to collect the information about used modules (Chapter~\ref{chap:dbltr}). 
One of the downsides of relying on dynamic traces is the high performance overhead of existing code instrumentation tools such as XDebug, which reportedly, can increase the page load time of web applications by up to 500\%~\cite{azad2019less}. 

\subsection{Symbolic Execution}
\label{sec:background_symbolic}
\begin{listing}[t]
	\begin{minted}[fontsize=\footnotesize, tabsize=2, linenos, startinline]{php}
		$user_name = $_POST['user'];
		if (!isset($username)) {
			$redirect_to = login_url('Username not provided.');
		}
		else {
			$user = get_user_by_login($user_name);
			if (!$user && strpos($user_name, '@') ) {
				$user = get_user_by_email($user_name);
			}
			if ($user) {
				$redirect_to = get_dashboard_url($user->ID);
			}
			else {
				$redirect_to = login_url('Invalid username.');
			}
		}
		wp_safe_redirect($redirect_to);
		exit();
	\end{minted}
	\caption{WordPress login routine. Successful login attempt requires a valid username or email address (line 6 and 8). Conversely, not providing the username or providing a non-existing username results in failed login (line 3 and 14).}
	\label{lst:wordpress_login}
\end{listing}

Symbolic execution is an offline program analysis technique that explores the reachability of different code branches by propagating symbolic values. 
For instance, for web applications, we mark user-provided values as symbolic. 
The symbolic placeholders for the user-controlled variables encompass all the possible values for these parameters. 
Throughout the analysis, we collect a set of constraints based on conditional operations and limit the set of feasible values for each symbolic variable. 
Upon encountering a branch with symbolic condition, we fork the execution and explore all feasible branches. 

Listing~\ref{lst:wordpress_login} shows a concise version of the login page of WordPress. 
By running the Selenium scripts from the Less is More dataset, which automate the interactions with common functions of WordPress, we would trigger the successful login via username (first column in Figure~\ref{fig:concolic_coverage}). 
In contrast, symbolic execution explores other paths within the same code leading to the inclusion of functions that handle failed login attempts (line 3, and 14) and login with email address (line 8) as depicted in Figure~\ref{fig:concolic_coverage}. 

\paragraph{Concolic execution:} 
Concolic execution combines concrete and symbolic execution. 
In this scheme, we replace symbolic variables with concrete counterparts depending on the use case. 
For instance, the transition from symbolic values can be used to generate concrete test cases that explore specific parts of an application~\cite{10.1145/1081706.1081750}. 

% \subsection{Challenges}
% %
% While symbolic execution provides a solution to model all possible executions
% of a program, it has its own challenges regarding the execution of real world
% and complex examples.
% %
% The challenges of symbolic execution are classified in four categories: 1)
% Memory, 2) Environment, 3) Space explosion, and 4) Constraint solving.
% %
% In the following, we will describe the challenges that rises for a symbolic
% engine.
% %

% %
% \subsubsection{State Space Explosion}
% %
% One of the main drawback of a symbolic engine is called the state space or path
% explosion problem.
% %
% Depending on the behavior of a symbolic engine, it can fork its execution of
% the program under test during loops, conditional branches, or function calls.
% %
% When a symbolic engine forks the execution of a program, the number of forked
% states can raise exponentially which ends in resource starvation which is
% called state space explosion (i.e., path explosion).
% %
% The major idea behind most solutions for path explosion is to reduce the number
% of states to be explored.
% %
% Each symbolic engine takes on different approach to achieve this.
% %
% Bounding the loop exploration to a limited number of iteration is one approach
% to reduce the number of states to explore.
% %
% An alternate approach is to discard states with un-satisfiable paths from the
% symbolic engines execution queue.
% %
% Prior to executing a forked path, if the constraint solver can prove that a
% logical formula given by the path constraints are not satisfiable, then
% symbolic engine can safely discard the execution.
% %

% %
% Symbolic engines also leverage function or loop summarization to tackle path
% explosion problem.
% %
% A symbolic engine can call a function \texttt{f} multiple times through the
% execution whether at same call-site or different ones.
% %
% Depending on the passed arguments or the call-site, the symbolic engine can
% reuse prior results of invoking the function \texttt{f} for the future
% executions prior to invoking the function.
% %

% %
% Another technique to prevent path explosion is called state merging, where a
% symbolic engine uses constraint solvers to merge multiple states into one in
% order to reduce number of states to be explored.
% %
% State merging has shown to be effective in decreasing the number of path to
% explore.
% %
% However, state merging heavily depends on the constraint solvers of symbolic
% engines which can hamper the performance of symbolic engines.
% %
% Furthermore, symbolic engines utilize different techniques such as taint
% analysis, fuzzing, branch predication, and type checking to prioritize
% different paths to explore.
% %
% Such techniques allow symbolic engines to focus on execution of more promising
% paths without causing path explosion.
% %
% %
% \subsubsection{Memory} 
% %
% Memory is one of the first challenges where a symbolic engine needs to specify
% how it handles complex objects in a programming language including arrays and
% pointers.
% %
% The memory model in a symbolic engine can heavily affect the achieved coverage
% on a program under test.
% %
% One the challenges of handling memory operation is called symbolic memory
% address problem~\cite{SurveySymExec-CSUR18}, where a referenced address in an
% operation in the program is a symbolic expression.
% %

% %
% Symbolic engines can use different level of memory generalization to tackle
% this challenge.
% %
% One can treat all memory addresses as fully symbolic values which is the
% highest level of generalization.
% %
% Symbolic engines take on different techniques in order to implement such
% generalization while accessing symbolic memory addresses such as state forking.
% %
% In state forking, a symbolic engine forks the state of execution on any read or
% write operation on symbolic addresses to consider all possible states of such
% operation.
% %
% An alternate approach is to model all possible memory manipulation in a
% application under test.
% %
% Such level of memory generalization often does not scale to complex
% application, since symbolic execution of a complicated memory operations can
% cause an explosion in number of possible states.
% %

% %
% \subsubsection{Environment}
% %
% The interaction of an application with the environment plays an important rule
% during its execution.
% %
% Most programs are not isolated from interacting wit the environment including
% the operating system such as file system and network.
% %
% As a result, the symbolic engine has to take into account the interaction of
% the application under test with the underlying operating system.
% %

% %
% There are multiple approaches to consider in order to handle environment
% interaction such as concrete execution or abstract modeling of interactions.
% %
% One approach is to perform any OS interaction with concrete arguments.
% %
% Note that the goal of a symbolic engine is to fully explore an application
% execution using symbolic values.
% %
% This solution limits the symbolic engine to fully explore the application since
% the interaction with OS is based on concrete values.
% %
% Another approach is to create an abstract model of the OS interactions such as
% generating symbolic files.
% %
% However, creating an abstract model for the library functions rather than the
% system call in an OS is expensive as the number of functions grows.
% %
% Depending on the application of the symbolic engine, these interactions can be
% handled differently.
% %
% For instance, automatic exploit generation tools models a partial of the system
% environment where an attacker can influence such as file systems and network
% sockets.
% %


% %
% \subsubsection{Constraint Solving}
% %
% There are many techniques that uses constraint solving in order to analyze,
% test, or verify an application.
% %
% For Satisfiability problems (SAT), constraint solvers takes a problem
% represented in a logical formula and determines whether there exist a set of
% boolean values for the symbols that makes the formula true.
% %
% Constraint solving allows the symbolic engine to determine the feasible
% solutions for the executed program, if the feasible solution exists.
% %
% However, constraint solving can be a major bottleneck for the scalability of
% symbolic engine, when it comes down to a complex combination of many
% constraints to solve.
% %
% Constraint solvers use different optimization to tackle the challenge of
% complex constraint such as expression rewriting and constant folding to reduce
% the number of constraints to solve.
% %
% An alternate solution is that constraint solvers use previous constraint
% solutions in order to speed up the constraint solving.
% %

% %
% \subsection{Log files}
% %
% In this subsection we explain how a web server records feedbacks of responding
% to user's requests.
% %
% Web servers such as Apache and Nginx provide a logging mechanism for the
% administrators to collect feedbacks on the activity and the performance of the
% server while responding to requests. \color{red} CITE \color{black}
% %
% The access-logs on a server records all incoming requests that a server
% processes.
% %
% The recorded information on an access-log differs depending on the specified
% format for the web server.
% %
% On a default configuration, when the web server receives a request, it records
% the IP of the machine sending the request, the time stamp, the type of request
% (i.e., \texttt{GET}, ...), the requested file, the response code, and the
% length of the response.
% %
% However, an administrator can customize the format of recorded access-logs
% using the configuration files of the web-server.
% %
% In our system, we extended the access-log formats to include the content of the
% cookies attached to the request.
% %
% Such implementation allows our system to concretize the symbolic value of
% cookies used in a web application on demand. 
% %

% %
