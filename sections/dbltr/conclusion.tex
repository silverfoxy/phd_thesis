\section{Summary}

In this chapter, we explored the idea of ``role-based debloating'', which consists of producing multiple versions of debloated web applications, each tailored to a cluster of users with similar usage behaviors (i.e., roles). 
We started by conducting a user study to understand how experienced developers and administrators interact with web applications. 
We then used this data in combination with \dbltr{}, our proposed tool that is capable of collecting code-coverage traces of the users of an application, to form clusters of users with similar usage behavior, and producing differently-debloated applications customized to the needs of each group. 
\dbltr{} also includes a content-delivery pipeline that can transparently route users to their clusters of dedicated debloated web applications without the need to modify the target web applications. 

Through our detailed analysis, we quantitatively showed that \dbltr{} can outperform the state-of-the-art in web application debloating. 
By incorporating the idea of role-based debloating, we can produce debloated web applications that are 30\% smaller in size, and contain 80\% fewer severe vulnerabilities (i.e., historic CVEs) compared to the ``globally'' debloated web applications produced by prior work. We also explored the contribution of each user to the code-coverage of roles, in an effort to understand the robustness of clustered debloating compared to the extreme where each user receives their own copy of the debloated applications. 

We showed that \dbltr{}'s clustering expands the code-coverage of similar features in each role for up to 38\% of all files, which affects more than half of the packages and classes in the web applications. This effect on the code-coverage allows \dbltr{} to retain the code for similar features that role members may use in future. 
Overall, our results demonstrate that role-based debloating is a superior approach compared to past global-debloating approaches, with tangible benefits, both in terms of security (greater degree of attack-surface reduction) as well as usability (lower likelihood of breakage and support for ``live'' introduction/migration of users into new and existing debloating clusters).
