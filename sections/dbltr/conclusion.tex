\section{Conclusion}

In this paper, we explored the idea of ``role-based debloating'', which consists of producing multiple versions of debloated web applications, each tailored to a cluster of users with similar usage behaviors (i.e., roles). 
We started by conducting a user study to understand how experienced developers and administrators interact with web applications. 
We then used this data in combination with \sys{}, our proposed tool that is capable of collecting code-coverage traces of the users of an application, to form clusters of users with similar usage behavior, and produce debloated applications customized to the needs of each cluster. 
\sys{} also includes a content-delivery pipeline that can transparently route users to their clusters of dedicated debloated web applications without the need to modify the target web applications. 

Through our detailed analysis, we quantitatively showed that \sys{} can outperform the state-of-the-art in web application debloating. 
By incorporating the idea of role-based debloating, we can produce debloated web applications that are 30\% smaller in size, and contain 80\% fewer severe vulnerabilities (i.e., historic CVEs) compared to the ``globally'' debloated web applications produced by prior work. We also explored the contribution of each user to the code-coverage of clusters, in an effort to understand the robustness of clustered debloating compared to the extreme where each users receives their own copy of the debloated applications. 

We showed that \sys{}'s clustering expands the code-coverage of similar features in each cluster for up to 38\% of all files, which affects more than half of the packages and classes in the web applications. This effect on the code-coverage allows \sys{} to retain the code for similar features that cluster members may use in future. 
Overall, our results indicate that role-based debloating is a practical approach with tangible security benefits, and should be considered as a means to protect security-critical web applications. 

