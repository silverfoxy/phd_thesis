\section{Experimental Design}\section{Experimental Design}
\label{sec:experiment-design}

\begin{figure}[t]
    \centering
    \includegraphics[]{figures/dbltr/EnvironmentPreparation.pdf}
    \caption{During the training phase, we collect user interaction code coverage traces to generate clusters. Based on these clusters, specialized debloated copies of web applications are produced for each cluster.}
    \label{fig:environment_preparation}
\end{figure}

In this section we lay the foundations and discuss the design decisions of our system. 
First we present the details of our user study and data collection. 
We then introduce the design concepts behind \dbltr{}, our automated pipeline for clustering users into debloating sets and assigning each set to a differently-debloated web application.
Figure~\ref{fig:environment_preparation} shows the high level steps including the data collection and clustering, followed by the preparation of the content delivery environment to serve debloated web applications.

\subsection{Usage patterns of web application features from the perspective of experienced administrators}

To understand how experienced users interact with web applications, we conduct the following user study. 
We hire our user-study participants by advertising paid projects on popular freelancing platforms, such as, Upwork~\cite{upwork}. 
After reviewing the resumes and interviewing the candidates, we hire experienced freelancers with 2-10 years of expertise on web development and system administration. 
We specifically interviewed candidates who mentioned phpMyAdmin, WordPress, or Magento on their resume. 
We focused on these web applications since they were used in the ``Less is More'' study~\cite{lessismore}, allowing us to compare and contrast our findings with it. 

The main goal of this user study is to understand whether administrators use different subsets of the overall features available to them. 
Moreover, through this study we analyze which features in the web applications in our dataset are commonly used among developers and administrators, and which features are relatively unpopular, used by only a fraction of administrators.

\subsubsection{User Study Tasks}

\begin{figure*}[h]
    \centering
    \includegraphics[width=0.95\columnwidth]{figures/dbltr/RoleModelsFlow.pdf}
    \caption{System Architecture of \dbltr{}. In Step 1, we provide the debloated web applications and user to cluster mappings to \dbltr{}'s content delivery module. User requests (2) are processed by \dbltr{}'s reverse-proxy (Step 3). After identifying the identity of the user (Steps 4-6), \dbltr{} internally routes the requests to the custom debloated web applications (Step 7).}
	\label{fig:system_architecture}
\end{figure*}

The task description for each project consists of an overview of the user study, the background required to participate, and the expected deliverables. 
Moreover, we included the information about the consent to participate in our study and described the information that we collect (i.e., server-side logs and code-coverage information). 

After interviewing the participants and reviewing their resumes, we hired 20 experts for each web application for a total of 60 experts on phpMyAdmin, WordPress, and Magento. 
We compensated the participants at the rate of \$15 per hour.
During the pilot experiments, we realized that not every freelancer is familiar with the concept of a user study. 
More importantly, to avoid future disputes, freelancers preferred to work on a predefined list of deliverables. 

Based on these observations, we defined two milestones for our user study. 
First, we asked our participants to provide a list of web application features that they commonly use in their daily tasks and projects. 
Most of our participants listed both maintenance and administration tasks. 
Among the common tasks, we observed verification of the functionality of the website (e.g., registering as new customers, submitting orders, etc.), maintenance tasks (e.g., backups, importing data, etc.) and even search-engine optimization. 

For the second milestone, we asked our participants to spend one hour of their time on our instrumented web applications and perform the tasks that they listed earlier. 
This process provided them with the list of deliverables and expectations, and also enabled us to validate their effort on this project. 

For freelancing platforms that provide a time-tracking utility, we use this feature to verify the participation of users together with cross-validating their task report with our code-coverage traces. 
For submissions that did not follow our guidelines (e.g., did not spend enough time, skipped the majority of tasks in the reports, etc.) we asked the participants to revise their submission. 

\textbf{IRB Approval} Since our experiments involved the assistance of real users, we obtained an Institutional Review Board (IRB) approval for our user study. 
Upon providing thorough details of our tasks and the human interactions, along with the information that we collect from the users, we obtained IRB approval on May 27, 2021. 

\subsubsection{Setup of Web Applications}

To facilitate the setup of web applications for our user study participants, we prepared the following environments:

\begin{itemize}
    \item \textbf{phpMyAdmin} (version 5.1.0), with multiple pre-populated databases including the ones from WordPress and Magento web applications.
    \item \textbf{WordPress} (version 5.8) with an admin account, over 20 blog posts, multiple pages, and comments. 
    \item \textbf{Magento} (version 2.3.5) configured with the ``Sample data'' package that includes an inventory of over 1,000 products. 
\end{itemize}

Each participant received their own instance with the admin credentials on a unique subdomain. 
We use a PHP code profiler to collect the usage traces from user interactions in the form of file- and line-coverage data.  

Throughout this user study, we interviewed over 110 individuals, some of whom decided not to participate in our study due to reasons such as non-recurring and short-term nature of our tasks, their busy schedule, etc.
Overall, we spent numerous weeks interviewing our participants and following up with them to ensure the timely delivery of their tasks. The cost of this experiment was approximately \$1,000, most of which was used to pay the administrators in our user study and the remainder to pay for domain names and the hosting of virtual machines on public clouds.

\subsection{Debloating Pipeline}

In this section we describe \dbltr{}, our proposed dynamic-debloating pipeline. 
\dbltr{} is able to identify web-application users with similar usage profiles (i.e., roles), and create debloated applications that are tailored to the needs of users in each profile. Note that \dbltr{}'s roles are completely independent of, and can work in conjunction with any RBAC roles that the debloated web application may have.


Unlike previous one-size-fits-all approaches, application users with specific usage patterns do not need to inherit the whole code-base from other users with vastly different usage behaviors. 
A concrete example of this effect becomes apparent in the usage traces of WordPress. 
Certain administrators focus on the content, and SEO. 
Therefore, they mostly interact with blog posts and page content, meta tags, and keywords. 
Conversely, another distinct group of administrators focused on changing the appearance of the website by installing new themes and plugins. 
In this scenario, by providing the ability to install new plugins to the first group, we are unnecessarily increasing their privileges beyond what they truly require. 

For certain web applications, their role-based access-control mechanisms can limit, to a certain extent, unnecessary capabilities.
Unfortunately many PHP applications that provide administrative features lack this ability (e.g., phpMyAdmin). 
Moreover, the provided roles do not always match the real-world requirements of users. For instance, WordPress comes with only six hardcoded roles ranging from ``Super Admin'' to ``Subscriber'' and the default role for new users is the most privileged one. 
The benefit of our approach is that we dynamically identify the required web-application features of different groups of users based on runtime traces of their past behavior. 

This benefit of \dbltr{} is most significant when applications do not provide fine-grained permission management. 
For instance, using the classic ``single debloated application for all users'' approach, an administrator would share the codebase with data-entry users or even unauthenticated public users. By clustering similar users together, we not only separate privileged from less-privileged ones on the same web application, but we also decrease the likelihood of breakage due to overly aggressive debloating, in the case of custom debloating for individual users.
We provide a quantitative analysis of this effect in Section~\ref{sec:augmented_coverage}.

\subsubsection{Producing the debloated web applications}

\dbltr{} extracts a list of features representing the usage profiles from the code-coverage traces. 
These features include file names, active namespaces, used classes, and invoked functions.  
These features are effective indicators of the functionality corresponding to each user's code-coverage. 
Most commonly, web application source files are partitioned under directories that indicate the feature they implement. 
Moreover, for external dependencies (i.e., composer packages), the file-path includes the name of the module that the files belong to. 
The same holds true for namespaces, class names and function names in that they usually represent the underlying feature that they implement. 

Nevertheless, \dbltr{} clustering does not rely on the naming scheme of the web application modules to be meaningful, as long as the naming scheme can uniquely represent the underlying feature that is being used. 
Based on these features, we cluster the code-coverage of users using unsupervised clustering algorithms, and optimize the number of clusters (i.e., roles) to produce the best debloating (i.e., highest number of removed functions across roles). 
\dbltr{} then produces the following artifacts:

\begin{itemize}
    \item $N$ debloated variants of web applications, where $N$ is the number of roles that optimizes the debloating by grouping users with similar behavior together. 
    \item Configuration files for the reverse-proxy and the web servers to host the debloated web applications in a containerized environment. 
    \item Database with the mapping of users to roles.
\end{itemize}

\subsubsection{Routing users to debloated web applications}

One of the design requirements of \dbltr{} is seamless user experience. 
Therefore, we incorporate a reverse-proxy that identifies user login information and routes any post-authentication traffic to specialized containers which include debloated web applications customized for that user. 

\dbltr{}'s reverse-proxy takes as input a configuration file for each web application. 
This file informs the proxy of how to identify successful login request-response pairs, the form field that contains the username, and the session cookie name to track the user past authentication. 

This information is stored along with the user to role mappings in an in-memory data store. 
For authenticated users, the proxy extracts the session cookie, queries the data store for the web server id for the user and routes the traffic to the appropriate debloated web application. 
From the user's perspective, there is no observable effect while this routing is taking place. Due to the architecture of \dbltr{}, two users can both be visiting the same URL at the same time, yet accessing radically different (i.e., debloated to match their needs) web applications on the server side.

\subsubsection{Sharing web application state between web server instances}

Each debloated web application instance is produced from the same source code. 
Nevertheless, the state of each web application depends on the database, cookies, and the session storage. 

In order to keep the state of all web applications synchronized, we connect all of them to an identical database instance in a containerized environment. 
Moreover, directories that include the session information are mounted as a single volume across all web application instances. 
By keeping the databases and session storage in sync across the web application copies, users that authenticate on one copy of the application and are redirected to another debloated instance will maintain their authenticated status. This separation of the data stores that keep track of the state of the web applications (i.e., databases and the session storage) from web servers is modeled after a key principle of scalable web applications allowing requests from the same user to be served by multiple web-server instances~\cite{scalability-book}.
Finally, since web browsers are responsible for maintaining valid cookies, after routing users to different debloated web application instances served over the same URL, user cookies are automatically transferred along with future requests, thereby providing a seamless experience for the users when the routing is taking place in the background.
