\section{Related Work} 
The idea of software debloating was initially discussed by Zeller et al.~\cite{zeller2002simplifying} as a means to isolate failure-inducing code. 
This idea was later applied to the context of software security to reduce the attack surface of applications. 
Ghavamnia et al. and Rastogi et al. explored the idea of debloating containers~\cite{rastogi2017cimplifier, 259711}, while Abubakar et al. debloated the kernel~\cite{abubakar2021shard}. 
Orthogonally, another line of research explores binary debloating~\cite{snyder2017most, redini2019b, heo2018effective, quach2018debloating, qian2020slimium, ghavamnia2020temporal, mishra2020saffire, koo2019configuration}, and debloating web applications~\cite{lessismore, saphire, mininode, jahanshahi2020you}. 

At a high level, there are three mainstream approaches to debloating: 
i) using static analysis to identify unreachable code~\cite{redini2019b, snyder2017most, quach2018debloating, mininode, 255308}, ii) 
debloating reachable code which is unused given a set of tests (e.g., automated test cases, or dynamic code-coverage traces)~\cite{lessismore, heo2018effective,qian2020slimium, koo2019configuration}, and finally, iii) API specialization, which consists of disabling sensitive APIs or hardening them with respect to the execution context of applications~\cite{mishra2020saffire, saphire, jahanshahi2020you, mishra2021sgxpecial}. 

Our work is mainly motivated by the ``Less is More'' approach of Amin Azad et al.~\cite{lessismore}. By comparing the debloating results of \sys{} with the baseline debloating approach of ``Less is More'' (Section~\ref{sec:debloatingresults}), we demonstrated that role-based debloating outperforms the ``Less is More'' approach, both in terms of security metrics, as well as on reducing concrete vulnerabilities. 

In another line of work targeting binaries and web applications, Mishra et al.~\cite{mishra2018shredder, mishra2021sgxpecial}, Bulekov et al.~\cite{saphire}, and Jahanshahi et al.~\cite{jahanshahi2020you} provided solutions to reduce the software attack surface of applications through limiting the list of available APIs for each piece of code. 
By incorporating their defense, attackers are limited in their ability to exploit the application vulnerabilities. 
These solutions are orthogonal to our work and can be used in combination with \sys{} to further protect the applications against the exploitation of potential vulnerabilities. 

Koishybayev et al. proposed Mininode, a tool to reduce the attack surface of Node.js applications by debloating third-party modules~\cite{mininode}. 
Their approach is based on static analysis which they use to identify unreachable code in third-party modules and the chain dependencies of Node.js applications. 
While static analysis is helpful in identifying unused code, several categories of common web application vulnerabilities (e.g., SQLi, XSS, CSRF, etc.) reside in reachable parts of the source code. 
\sys{} incorporates dynamic analysis and therefore, is capable of debloating even the reachable but unused parts of the code. 

Bocic et al. and Son et al. studied access-control bugs in web applications~\cite{7582754, son2013fix}. 
They analyzed access-control flaws in Ruby on Rails and PHP applications and identified over 100 authorization bugs in the existing open-source applications. Their findings reinforce the motivation for \sys{}'s role-based debloating which can guarantee the separation of public vs. authenticated users, even in the presence of access-control errors.