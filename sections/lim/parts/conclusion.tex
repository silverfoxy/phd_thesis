\section{Conclusion}
%With the evolution of the Internet, web applications became more and more complex over time to offer better and richer experiences to users.
%Simple applications that would deliver static pages before the turn of the century turned into dynamic and feature-rich applications that span thousands of files and lines of code.
%This evolution also introduced new development paradigms like heavy code reuse through package managers.
%Instead of coding a whole application from scratch, it is faster and easier for developers to rely on external packages that are properly maintained and tested in order to focus of the core of an application.
%Overall, a whole ecosystem of languages and tools has seen the light of day to sustain the fast-paced evolution of the Internet and the applications that run on it.
%However, this development brought with it its fair share of security issues.
%As web applications are reaching very large sizes with complex structures and data flows, it opens the door to a trove of vulnerabilities that is hard to prevent.

In this paper, we analyzed the impact of removing unnecessary code in modern
web applications through a process called \textit{software debloating}.
We presented the pipeline details of the end-to-end, modular debloating
framework that we designed and implemented, allowing us to record how a
PHP application is used and what server-side code is triggered as a result of
client-side requests. After retrieving code-coverage information, our debloating
framework removes unused parts of
an application using file-level and function-level debloating.

By evaluating
our framework on four popular PHP applications (phpMyAdmin, MediaWiki,
Magento, and WordPress) we witnessed the clear security benefits of debloating web
applications. We observed a significant LLOC decrease ranging between
9\% to 64\% for file-level debloating and up to an additional 24\% with
function-level debloating. Next, we showed that external packages are one
of the primary source of bloat as our debloating framework was able to remove
more than 84\% of unused code in versions that used Composer, PHP's most popular
package manager. By quantifying the removal of code associated with critical
CVEs, we observed a reduction of up to 60\% of high-impact, historical vulnerabilities.
Finally, we showed that the process of debloating also removes
instructions and classes that are the primary sources for attackers to build
gadgets and perform POI attacks.

Our results demonstrate that debloating web applications
provides tangible security benefits and therefore should be seriously
considered as a practical way of reducing the attack surface of
web-applications deployments.

\vspace{0.5ex}
\noindent \textbf{Acknowledgements:} We thank our shepherd Giancarlo Pellegrino
and the anonymous reviewers for their helpful feedback.  This work was
supported by the Office of Naval Research (ONR) under grants N00014-16-1-2264
and N00014-17-1-2541, as well as by the National Science Foundation (NSF)
under grants CNS-1813974 and CMMI-1842020.

