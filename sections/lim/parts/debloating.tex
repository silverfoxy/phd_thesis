\section{Debloating web applications}
\label{sec:debloating}

In this section, we briefly describe the evaluated debloating strategies and
the steps we took to ensure that the debloated applications remain functional.


\subsection{Debloating strategies}

By combining the simulated usage of a web application (achieved through
tutorials encoded in Selenium scripts, web crawlers, monkey testing, and
vulnerability scanning) with
server-side code profiling, we can identify the code that was executed
as part of handling web requests. Consequently, code whose execution was
not triggered by any client-side request can presumably be removed since
it is not necessary for any of the functionality that is desired by users
(as quantified by the utilized usage profiles).
In this work, we evaluate the following debloating strategies:

\begin{itemize}
    \item \textbf{File-level debloating:} Given that the source code of web applications spans tens or hundreds of different files, we can completely remove a file, when none of the lines of code in that file were executed during the stimulation of the web application.
    \item \textbf{Function-level debloating:} In function-level debloating, not only can we remove entire files, but we can also selectively remove some of the functions contained in other files. This is a more fine-grained approach which allows us to remove more code, than the more conservative, file-level debloating strategy.
\end{itemize}

More fine-grained approaches are possible, such as, the removal of specific code statements from retained functions which were not exercised during stimulation.
However, such changes essentially modify the logic of a function (e.g., removing
conditional code blocks) thereby increasing the probability of breaking the
resulting program when a minute change of a client-side request would lead the
execution into these blocks of code.

\subsection{Detecting the execution of removed code}
We replace all removed functions and files with placeholders which, if executed,
have the following tasks:
\begin{itemize}
    \item \textbf{Exit the application:} If a placeholder happens to be triggered, the PHP application will start its shutdown procedures. This way, the application does not enter an unexpected state that was not planned by the debloating process.
    \item \textbf{Record information about the missing function:} In order to better understand which missing placeholders were triggered and how, our framework logs several pieces of information, such as, the URL that triggered the execution of the removed code, the name of the class and function of the removed code, and the corresponding line numbers.
\end{itemize}

To ensure that the debloating process has preserved the functionality of
the debloated web application, we rerun all the Selenium-mapped tutorials and monkey scripts
after the debloating stage. If our placeholder code for removed files and
functions executes during this stage, this means that this code should not
have been removed.

This feedback mechanism proved invaluable during the development of
our framework since it helped us identify problems with our coverage
logic which in turn revealed the challenges that we described in
Section~\ref{subsubsec:challenges}.