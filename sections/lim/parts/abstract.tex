\subsection*{Abstract}

As software becomes increasingly complex, its attack surface expands enabling
the exploitation of a wide range of vulnerabilities. Web applications are no
exception since modern HTML5 standards and the ever-increasing capabilities
of JavaScript are utilized to build rich web applications, often subsuming
the need for traditional desktop applications. One possible way of handling
this increased complexity is through the process of software debloating, i.e.,
the removal not only of dead code but also of code corresponding to features
that a specific set of users do not require. Even though debloating has been
successfully applied on operating systems, libraries, and compiled programs,
its applicability on web applications has not yet been investigated.

In this paper, we present the first analysis of the security benefits of
debloating web applications. We focus on four popular PHP applications and
we dynamically exercise them to obtain information about the server-side code
that executes as a result of client-side requests. We evaluate two different
debloating strategies (file-level debloating and function-level debloating)
and we show that we can produce functional web applications that are 46\%
smaller than their original versions and exhibit half their original cyclomatic
complexity. Moreover, our results show that the process of debloating removes
code associated with tens of historical vulnerabilities and further shrinks
a web application's attack surface by removing unnecessary external packages
and abusable PHP gadgets.

%Use of rich web applications is becoming more widespread these days, as we move our complex systems to the web, we face different types of attack vectors and complex systems are harder to test, debug and reason about. One solid line of defense against exploitation has always been attack surface reduction, by limiting access to unnecessary services we can potentially prevent attackers from exploiting vulnerabilities in them. In this paper, we talk about a concept well known in binary world but less studied in the context of web. Software debloating is the process of removing code that is not necessary, e.g., instrumented by specific subset of users.

%First, we analyze four popular PHP web applications* top map known vulnerabilities to parts of code that results them, the chosen web applications span over four main web application categories namely, Administration tools, Wikis, CMSs and Online Shops. Second step is to dynamically analyze the applications to see if vulnerable lines of code are fired during user interaction. To address that, we create automation script to perform the tasks mentioned in online tutorials of those applications. To come up with a representative profile of general users of each application we enhance the code coverage from tutorials with the coverage from spiders that crawl web applications and monkey tests that try to brute force every possible action within the application. Third, based on the coverage resulted from tests in step two, we measure the coverage of code paths that intersect with known vulnerable lines. Arguably, if most users of an application do not use a feature, that feature can be removed, hence, removing the vulnerabilities within it.

%\todo{(Finalize numbers)} Our results show that on average we can remove \%75 of vulnerabilities by removing functions that do not fire during general use of users.  Debloating strategies can vary by their level of aggression and how far we want to go in terms of removing code from our application. By removing functions from studied web applications that would not be used by users of an application, we can significantly reduce the attack surface. In addition to the results, we open source our framework to record the coverage of individual lines within a web application under different usage profiles and automatically debloat the application under file and function level debloating conditions.
