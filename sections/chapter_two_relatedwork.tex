\chapter{Related Work}
\label{chap:relatedwork}

In this chapter we review the literature of research on the topic of software debloating. 
The idea of software debloating was initially discussed by Zeller et al.~\cite{zeller2002simplifying} as a means to isolate failure-inducing code. 
This idea was later applied to the context of software security to reduce the attack surface of applications. 
At a high level, there are three mainstream approaches to debloating: 

\begin{enumerate}
    \item Using static analysis to identify and debloat unreachable code (i.e., dead code)~\cite{redini2019b, snyder2017most, quach2018debloating, mininode, 255308}.
    \item Debloating reachable code which is unused given a set of tests (e.g., automated test cases, or dynamic code coverage traces)~\cite{lessismore, heo2018effective,qian2020slimium, koo2019configuration}.
    \item API specialization, which consists of disabling sensitive APIs or hardening them with respect to the execution context of applications~\cite{mishra2020saffire, saphire, jahanshahi2020you, mishra2021sgxpecial}. 
\end{enumerate}

Orthogonally, we can categorize debloating methods based on their target platform. 

\section{Debloating for the web}

Web applications consist of client side and server side modules. 
The client side modules, mainly including HTML, JavaScript and CSS, directly interact with clients' browsers. 
Snyder et al. investigated the idea of API specialization for the JavaScript APIs in web browsers~\cite{snyder2017vibrate}. 
The authors performed a risk analysis of providing access to various browser features and APIs to the JavaScript code.
They evaluated the use of different
JavaScript APIs in the wild and proposed the use of a client-side extension
which controls which APIs any given website would get access to, depending
on that website's level of trust. 
Schwarz et al. similarly utilized a browser
extension to limit the attack surface of Chrome and show that they are able
to protect users against micro-architectural and side-channel
attacks~\cite{Schwarz2018}. 
Finally, Qian et al. debloated the Chromium browser with respect to the feature usage of specific websites~\cite{qian2020slimium}. 
These studies are orthogonal to our work in this thesis since
they both focus on the client-side of the web platform, whereas we focus on
the server-side web applications.

Web servers, as the integral part of serving web applications, are prime candidates for debloating. 
Koo et al. explored the debloating of web servers through the analysis of unused features based on a given web server configuration file~\cite{koo2019configuration}. 
Ghavamnia et al. identified that applications such as web servers by design have separate phases (e.g., setup vs serve), and introduced the idea of temporal system call specialization to limit the available APIs based on the current status of the target application. 

For the server-side debloating, Boomsma et al. performed dynamic profiling of a custom web application (a PHP application from an industry partner)~\cite{boomsma2012Dead}. 
The authors measured the time it takes for their dynamic profile system to get
complete coverage and the percentage of files that they could remove. Since the
application was a custom one, the authors were not able to report specifics
in terms of the reduction of the programs attack surface, as that relates
to CVEs. 

Koishybayev et al. focused on the bloat from third-party Node.js libraries. 
The authors designed Mininode, which uses static analysis to identify unreachable code in third-party modules and the chain of dependencies in Node.js applications~\cite{mininode}. 
The threat model of Mininode assumes the presence of a arbitrary code execution vulnerability and prevents the attackers from escalating their code execution privileges by accessing sensitive APIs within the main application dependencies. 
Nevertheless, their threat model does not cover the majority of web vulnerabilities including SQLi~\cite{sqlInjection}, XSS~\cite{xss}, CSRF~\cite{csrf}, etc. which by definition are only exploitable from the reachable code. 

Another group of researchers focused on API specialization to protect web applications against SQL injection attacks~\cite{jahanshahi2020you}.
Orthogonally, Redini et al. proposed BreakApp, which protects Node.js applications by limiting the available APIs for Node.js libraries~\cite{vasilakis2018breakapp} and Jahanshahi et al. designed Saphire, which is their API specialization approach that functions by identifying and limiting the system calls available to each PHP script, thereby limiting the ability of attackers in causing harm~\cite{saphire}. 
Their threat model is similar to Minonode and protects web applications from further exploitation in the presence of a vulnerability that leads to arbitrary code execution. 
API specialization is orthogonal to the debloating schemes discussed in this thesis and can be applied in parallel to our work to provide further protection for web applications. 


\section{Debloating in other platforms}

The body of research on debloating binaries consists of mechanisms that operate on the source code level, and those that debloat libraries and system calls. 
From the former category, Regehr et al. developed \textit{C-Reduce} which is a tool that performs reduction of C/C++ files by applying very specific program transformation rules~\cite{regehr2012CReduce}.
%Perses
Sun et al. designed a framework called \textit{Perses} that utilizes the grammar of any programming language to guide reduction~\cite{sun2018perses}.
Its advantage is that it does not generate syntactically invalid variants during reduction so that the whole process is made faster.
%Chisel
Heo et al. worked on \textit{Chisel} whose distinguishing feature is that it performs fine-grained debloating by removing code even on the functions that are executed, using reinforcement learning to identify the best reduced program~\cite{heo2018effective}.
%Summary

All three aforementioned approaches are founded on Delta debugging~\cite{zeller2002Delta}.
They reduce the size of an application progressively and verify at each step if the created variant still satisfies the desired properties (e.g., successfully compiles, or passes a set of predefined test cases).

Contrastingly, Sharif et al. proposed \textit{Trimmer}, a system that goes further than simple static analysis~\cite{sharif2018Trimmer}.
It propagates the constants that are defined in program arguments and configuration files so that it can remove code that is not used in that particular execution context.
However, their system is not particularly well suited for web applications where we remove complete features.

Redini et al. utilized abstract interpretation to identify basic blocks within the code that are unused~\cite{redini2019b}. 
They start by building the control flow graph of their target application and removing the non-connected nodes (i.e., basic blocks). 
Mishra et al. explored ways to identify and build profiles of allow-lists for system calls and the parameters passed to them~\cite{mishra2018shredder,mishra2020saffire}. 
Their threat model includes attackers who abuse code execution exploits and limits their ability to abuse the existing code from the applications and disrupts the gadget chains. 

Debloating Kernels, containers and trusted execution environments have also received the attention of researchers~\cite{abubakar2021shard,mishra2021sgxpecial}. 
Rastogi et al. looked at debloating a container by partitioning it into smaller and more secure ones~\cite{rastogi2017Cimplifier}. They perform dynamic analysis on system-call logs to determine which components and executables are used in a container, in order to keep them. 
Ghavamnia et al. looked at this problem from another perspective. 
They proposed Confine, their tool generates system call profiles for containers~\cite{259711}.
Lastly, Abubakar et al. debloated the Linux kernel~\cite{abubakar2021shard} and Mishra et al. proposed an API specialization scheme for Intel SGX APIs~\cite{mishra2021sgxpecial}.

\section{Control flow graph generation}
CFGs (Control Flow Graphs) are useful representations of the control flow of the applications, and are commonly used for static analysis. 
CFGs are also proven to be useful for the purposes of debloating. 
Redini et al. in their tool Bintrimmer showed that given an accurate CFG of a binary, they can identify disconnected nodes and remove them via debloating~\cite{redini2019b}. 
For web applications specifically, combining CFGs with the information about the entry points of applications is enough to perform debloating. 

The main challenge of CFG generation specially for applications written in dynamic languages such as PHP is the absence of critical information for static analysis. 
More concretely, PHP allows developers to dynamically decide on which modules to load, and call functions with dynamic names. 
Luckily, a full CFG is not a hard requirement for static analysis purpose built for vulnerability discovery. 

As a result, previous work has addressed the uncertainty of static CFG generation through dynamic analysis. 
For instance, Alhuzaili et al. and Jensen et al. incorporated web crawlers as a pre-analysis step to exercise various parts of the web applications, and as a result collect dynamic traces of the file inclusions and function calls~\cite{alhuzali2018navex, jensen2012thaps}. 
This way, they can aid their static analysis to build a more accurate CFG. 
Nevertheless, crawlers by design cannot explore all the possible paths within an application. 
Specially for complex applications with multi-stage form submissions, file uploads and form field verifications, web crawlers fall short. 

Orthogonally, Backes et al. proposed the idea of Code Property Graphs (CPGs) in which they use to detect PHP vulnerabilities. 
CPGs are generated based on CFGs and CGs (Call Graphs). 
In their work, the authors reported that they could only resolve 78.9\% of dynamic function calls. 
While the authors were able to use the CPGs to report new vulnerabilities, using a CFG with 20\% uncertainty makes debloating infeasible. 

An alternative solution devised by researchers to resolve dynamic function calls and file inclusions is to rely on the existing structure of these calls which are commonly made up of multiple string concatenations. 
For instance, Bulekov et al. modeled the file inclusions as regular expressions and mapped them to the underlying file system to identify potential candidate files~\cite{saphire}. 
Similarly, Dahse et al. in their tool RIPS, which incorporates static taint analysis to identify injection style vulnerabilities~\cite{dahse2010rips}, relied on a limited scope variable analysis to resolve the strings used in file inclusions. 
Pixy, one of the first papers to incorporate taint analysis to detect XSS vulnerabilities functions the same way~\cite{jovanovic2006pixy}. 
Overall, we observe that the previous research relies on incomplete CFGs, yet it allows their static analysis tools to identify new vulnerabilities in web applications.

For debloating purposes, an unresolved call site or file inclusion in CFGs leads to error-prone debloating. 
A call site that can potentially call any other function within the application means that debloating cannot safely reason about and remove any functions from the web application, and is therefore rendered useless. 

\section{Symbolic execution}

The idea of symbolic execution for program testing was originally introduced by King et al.~\cite{king1976symbolic} in 1975. 
The premise of symbolic execution is to explore the various execution paths of an application without concretely running it. 
Symbolic execution can be applied on the binaries (i.e., IR-Less) as well as higher level representations of a program (i.e., IR-Based) such as the LLVM IR~\cite{llvmir}. 
KLEE~\cite{cadar2008klee}, S2E~\cite{chipounov2009selective}, and angr~\cite{cheng2016binary} are IR-based while QSYM~\cite{yun2018qsym}, and SAGE~\cite{godefroid2012sage} are IR-less symbolic analysis engines. 
Generally, IR-based engines are easier to implement as they only need to support the fewer IR instructions, and are architecture agnostic. 
Conversely, IR-less implementations are harder to implement and need to support thousands of instructions and are closely tied to the underlying architecture that they support. 
At the same time, IR-less symbolic execution engines provide a better performance~\cite{poeplau2019systematic}.
Symbolic execution engines commonly consist of a symbolic store to include the current variables and their value set and its constraints. 
SMT solvers to evaluate the satisfiability of path conditions~\cite{moura2008z3}. 
State managers to hold the current state of conditions and constraints of the undergoing evaluation along with a set of path constraints. 
By iterating through the instructions of a program, the symbolic execution engine updates the aforementioned structures. 
For instance, if the engine is analyzing a code inside a branch statement, it assumes that the condition of the branch is already satisfied. 

Symbolic execution can be used to automatically generate concrete test cases, as proposed by DART~\cite{sen2009dart, sen2005cute}. 
Alternatively, symbolic execution is proven to be useful for vulnerability analysis (e.g., whether a certain path to sensitive program sinks is satisfiable)~\cite{5504701, wang2009intscope, cha2012unleashing, cadar2008klee}. 
Fuzzing tools also benefit from symbolic execution~\cite{godefroid2012sage}.
More interestingly, constraint solvers and variations of symbolic execution has been used for automatic exploit generation (i.e., generating an input that satisfies the path constraints to reach and exploit a known vulnerability)~\cite{alhuzali2018navex, avgerinos2014automatic}.

Concolic execution refers to the interoperation symbolic analysis and concrete execution. 
In this method, programs are executed with concrete values and parameters from unknown sources (e.g., user input, and the environment) are replaced by symbolic values. 
Concolic execution engines commonly rely on SMT solvers to generate concrete inputs that satisfy the existing path constraints. 
By replacing symbolic values with concrete ones, we reduce the overhead and increase the performance of the analysis. 

In the realm of web applications, Navex used concolic execution as a means for automatic exploit generation~\cite{alhuzali2018navex}. 
In a worker closer to ours, Naderi et al. built PHP emulators and used them in combination with counterfactual execution to uncover obfuscated PHP malware~\cite{naderi2019cubismo,naderi2019malmax}. 
Counterfactual execution is a method of program exploration under which all the loops are unrolled and branches are explored regardless of the satisfiability of their conditions. 

At a high level, symbolic execution challenges can be categorized as follows:

\begin{itemize}
    \item \textbf{State explosion:} Refers to the exponential increase in the number of branches to be explored. This can quickly lead to memory exhaustion, or lack of enough coverage before timeout.
    \item \textbf{Imprecise analysis:} Due to incomplete modeling of the underlying execution environment, the analysis can produce too many false positives or false negatives, or even invalid results. 
    \item \textbf{Constraint solving:} Is a costly task and can lead to timeouts, slowdowns or even unfeasible constraints.
\end{itemize}

To address the main challenge for symbolic execution which is the state explosion, previous work has incorporated methods such as pruning unrealizable paths~\cite{schwartz2015conflict}, function and loop summarization~\cite{boonstoppel2008rwset, godefroid2007compositional, godefroid2011automatic, xie2016proteus}, and state merging~\cite{godefroid2007compositional, hansen2009state}. 
We incorporate a subset of these mechanisms in our work, which we discuss later in Chapter~\ref{chap:ad} and Chapter~\ref{chap:conclusion} under Section~\ref{sec:futurework}. 
In summary, we devise code coverage based heuristics to prioritize paths that lead to new code coverage to address the stat explosion. 
We conduct an iterative debugging procedure to address the correct and precise implementation of commonly used PHP features. 
Finally, we argue that our debloating scheme based on concolic execution benefits from a loose constraint solving without requiring full SMT solvers.
